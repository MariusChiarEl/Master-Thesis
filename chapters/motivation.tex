\chapter*{Motivation} 
\addcontentsline{toc}{chapter}{Motivation}

The blockchain technology \cite{blockchain} is relatively new if we consider the release of the Bitcoin cryptocurrency blockchain newtork, which took place in 2009. However, its foundation was formed decades ago, in 1989, when Leslie Lamport developed the Paxos protocol \cite{paxos}, a distributed consensus mechanism that became the base of many decentralized systems' implementation.

While it may be primarily associated to cryptocurrencies, blockchain has grown broader in terms of its applications. Its infrastructure's main qualities are the decentralized trust, transparency and immutability, which make it perfect in domains such as electronic voting, digital identity or finance. However, one of its advantages is also a critical shortcoming, because, once deployed, a smart contract cannot be modified, leaving the door open to attackers if there is any vulnerability to exploit, which is almost impossible to avoid completely.

This thesis' motivation is to address the most vulnerable sector of the blockchain community, the begginer developers and students. We propose a desktop application that integrates Slither, a fast and powerful security analysis tool, and GitHub's REST API to automatically scan smart contracts and provide an overall security report and a visual guide towards the vulnerabilities. This application should act as a training wheel for the begginer developers, but it can also be paired with a thorough manual analysis for the more advanced ones.

We hope that this kind of tools will improve the security and efficiency of smart contracts for the long term, which will prevent unnecessary electricity consumption and funds losses. While no security analysis tool is perfect and can miss certain vulnerabilities, it can prevent obvious and high-impact attacks from taking place and alleviate the developers' pressure to deliver secure smart contracts.