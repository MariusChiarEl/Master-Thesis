\chapter*{Concluzii} 
\addcontentsline{toc}{chapter}{Concluzii}

This thesis gravitated around three main subjects: blockchain's infrastructure, Slither's design and perfromance analysis and SmartScan, the application made to ease the security analysis process. Now we have a better picture of what blockchain is and how it functions, including qualities and shortcomings alike, potential attacks that affect this kind of networks and the currently available solutions to prevent them. The second chapter was dedicated to Slither, one of the best tools for security analysis, with emphasys on its infrastructure, functionality and performance compared to similar solutions. Lastly, we have introduced SmartScan, the application that integrates Slither to automatically detect vulnerabilities in remote projects. The description of SmartScan includes its infrastructure, feature set, workflow and examples of security reports generated by it. Our hope is that our contribution will enable aspiring developers to create secure smart contracts and that the blockchain will be safer to use in the long term as a consequence.

In conclusion, the blockchain is a very promising technology, with essential applications in various domains that evenly distributes the power and responsability to all its users. However, its security concerns keep its popularity low among most people and more security measures must be taken in order to restore its image and make it usable in key areas, such as electronic voting, which could significantly improve the voting process by easing it for both the citizens and the states. Nevertheless, there is a long road ahead of blockchain towards achieving its potential and it will need a lot of support from the scientific community for that to happen, much like any other new piece of techonology.

%chapter 1
%- scurta descriere blockchain, ce face bine si ce nu
%- cateva atacuri asupra blockchain
%- ce unelte exista pentru analiza securitatii si care s cele mai populare
%- ce tipuri de astfel de unelte exista si cum functioneaza

%chapter 2
%- ce este Slither si cum functioneaza
%- performana lui Slither in diverse scenarii
%- detectia e optimizari a lui Slither (mentionata)
%- consideratii etice in folosirea Slither

%chapter 3
%- ce este si cum functioneaza SmartScan
%- ce aduce in plus
%- care ar fi fluxul e lucru
%- componentele din care e format
%- sistemul de rating al securitatii proiectului
%- exemplu pentru functiile implementate
%- exemplu de raport de securitate pe 3 proiecte asemanatoare
%- cum poate fi folosit si unde ar excela
%- idei de imbunatatiri