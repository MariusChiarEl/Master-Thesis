\begin{abstract}
    The Blockchain, despite its popularity and presence on the internet and in the financial sector, is a relatively new technology and this fact is proved by the large amount of attacks that took place on such networks. This situation raises the need for secure and robust networks that can keep the clients’ financial assets safe. Fortunately, there are various solutions to this category of problems, like Slither or Manticore, which take different approaches towards finding vulnerabilities. For the moment, most of these tools provide an at least decent detection for a high variety of attacks, but they need to get used more in order to ensure a higher standard of security when it comes to smart contracts. A good way to achieve this is to integrate the scanning tools into user-friendly applications that shorten the process of analyzing and fixing vulnerabilities.
\end{abstract}

\chapter*{Introduction} 
\addcontentsline{toc}{chapter}{Introduction}

The 21st century, so far, featured many advancements in the technology field. In the first decade, the internet had its peak spread among the high majority of people, the second one had seen a large grouth of the smartphone's adoption across the globe and now, in the third decade, the artificial intelligence is the most used and discussed breakthrough. The blockchain, while more nieche than the other technologies, also gained very much traction in the last 10 years, especially since 2020, when many people tried to take advantage of the sudden rise of Bitcoin, in both popularity and value, through mining or trading.

The blockchain is a decentralized, distributed and immutable ledger in which every user can add new blocks without the need of a central authority. Its foundation as a concept can be traced back to Leslie Lamport's Paxos protocol \cite{paxos}, released in 1998, which describes a system designed to achieve consensus among a network of computers, which is needed for agreeing on each iteration of the ledger, as it cannot be modified after its submission.

These systems, starting with the release of Ethereum, allowed the development of smart contracts \cite{smartContracts}, which are self-executing applications stored on the blockchain. The instructions of a smart contract are triggered when certain conditions are met and can be called without the need of a server. This property, along with its deterministic, transparent and immutable nature, gives the user the most power and responsability possible.

This technology has emerged only about 16 years ago, which can be seen because of the multitude of attacks that affected the networks that are based on it. This is why there is a need for secure and rezillient networks that can prevent the loss of clients' assets. Thankfully, there are many different tools that alleviate this concern, such as Slither \cite{slither} and Manticore \cite{manticore}, which find vulnerabilities in smart contracts in different ways. The aim of this thesis is to show Slither's strenghts and introduce SmartScan, an open-source desktop application that integrates it in a more seamless process of detecting and fixing security errors. This application is especially meant for students and aspiring smart contract developers.